\chapter{Conclusion}
\renewcommand{\baselinestretch}{\mystretch}
\label{chap:Con}
\PARstart{I}{n} conclusion, this project successfully applied the fuzz testing to test open-source HST. Not only all core objectives but also 80\% of extensive objective were achieved. Detected bugs were analysed through the qualitative and quantitative method. The result shows that ABC is a reliable HST which has a strict limitation on Verilog input. Designed grey-box fuzzer reaches 90\% probability of finding at least one bug. The generation time of this fuzzer is in a reasonable range. The designed tool-flow is useful and well-structured.

With further polishing, the designed testing tool can be promoted as a distributed Linux software in a collaborative public manner. The measured data and results were reliable through multiple measurements and averaged values. Following section discuss the future improvement of this project.


\section{Future work}
This project can be improved in future analysis: 
\begin{itemize}
\item Bug injection technology \cite{calagar2014source} could be involved to evaluate the bug-finding ability of this project. For example, the synthesis process of ABC can be manipulated to yield a netlist with bugs. This buggy netlist could be used to check whether the injected bug can be detected by our tool. 
    \item Fuzzing algorithm can be improved by deep learning and integrated with a coverage-guided feedback loop. Williams developed ''Covered'' is a Verilog code coverage detecting tool and it could be used to in the feedback loop.   
    \item Big-data can be combined with this project in bug identification and analysis.
    \item Automatic correction of the detected bugs. This could refer to Fujita's research of the automatic correction of logical bugs \cite{fujita2018automatic}.
\end{itemize} 
